\chapter*{Abstract}
\addcontentsline{toc}{chapter}{Abstract}

Chúng tôi giới thiệu Graph Attention Network (GAT), mô hình mạng nơ-ron mới hoạt động trên dữ liệu dạng đồ thị, sử dụng các lớp tự quan tâm đã đặt mặt nạ để cải thiện những nhược điểm của các phương pháp trước đó dựa trên hỗn hợp đồ thị hoặc các ước lượng của chúng. Bằng cách xếp lớp trong đó các nút có thể quan tâm đến các đặc trưng của lân cận, chúng ta cho phép (ẩn dụ) chỉ định các trọng số khác nhau cho các nút trong một lân cận, mà không cần bất kỳ hoạt động ma trận chi phí (như đảo ngược) hoặc phụ thuộc vào việc biết cấu trúc đồ thị trước.
Theo cách này, chúng tôi giải quyết một số vấn đề chính của các mạng nơ-ron đồ thị dựa trên tần số cùng lúc, và làm cho mô hình của chúng tôi dễ dàng áp dụng cho cả vấn đề dẫn chứng và dẫn điểm. Mô hình GAT của chúng tôi đã cho kết quả ??? trên bốn mô hình kiểm thử transductive và inductive: các tập dữ liệu Cora, Citeseer và Pubmed, cùng với tập dữ liệu tương tác protein-protein (ở đây, test graph bị ẩn trong quá trình học máy).


\vspace{8pt}
\noindent \textit{\textbf{Keywords}: Graph Attention Networks(GAT), transductive, Vietnamese multi-aspect dataset.}

