\chapter{Giới thiệu}
\label{chap:Giới thiệu}

% -------------------------------------------------------------------
% Motivation
% -------------------------------------------------------------------

Mạng Nơ-ron Tích Chập (Convolutional Neural Networks - CNN) hiện đang được áp dụng để giải các bài toán phân loại hình ảnh (He et al., 2016), phân vùng ngữ nghĩa ảnh (Jegou et al., 2017) hoặc dịch máy (Gehring et al., 2016), trong đó dữ liệu nền được biểu diễn dưới cấu trúc giống mạng lưới. Các mô hình này tái sử dụng hiệu quả các bộ lọc cục bộ, với các tham số học, bằng cách áp dụng chúng cho tất cả các vị trí đầu vào.
Tuy nhiên, nhiều tác vụ có liên quan đến dữ liệu không thể biểu diễn dưới dạng mạng lưới dưới dạng bất quy tắc. Bản dựng hình 3D, mạng xã hội, mạng viễn thông, mạng sinh học hoặc mạng kết nối não bộ là các ví dụ điển hình. Những loại dữ liệu này thường được biểu diễn dưới dạng đồ thị.
Trong lĩnh vực công nghệ thông tin, đã có một số nghiên cứu về việc mở rộng mạng nơ-ron để xử lý các đồ thị . Các nghiên cứu ban đầu sử dụng mạng nơ–ron đệ quy để xử lý dữ liệu được biểu diễn dưới dạng đồ thị đường vòng được định hướng (Frasconi et al., 1998; Sperduti \& Starita, 1997). Mạng Nơ-ron Đồ Thị (Graph Neural Network - GNN) được giới thiệu trong Gori et al. (2005) và Scarselli et al. (2009) là một mô hình chuẩn hóa của mạng nơ-ron đệ quy có thể trực tiếp xử lý các loại đồ thị bao quát hơn, ví dụ như đồ thị có chu trình, có hướng và không hướng. GNN bao gồm một quá trình lặp đi lặp lại, truyền trạng thái của nút cho đến khi đạt đối tượng; sau đó là một mạng nơ-ron, tạo ra một kết quả cho mỗi nút dựa trên trạng thái của nó. Ý tưởng này đã được Li et al. (2016) thừa hưởng và cải tiến bằng cách sử dụng đơn vị gated recurrent (Cho et al., 2014) trong bước truyền.

Trong khi đó, có một sự quan tâm tăng lên về việc mở rộng bộ lọc tổng quát cho không gian đồ thị. Những tiến bộ trong hướng này thường được phân loại thành các phương pháp phần tử và phương pháp không phần tử. Trên một bên, các phương pháp phần tử làm việc với một biểu diễn phần tử của đồ thị và đã được áp dụng thành công trong mục tiêu phân loại nút. Trong Bruna et al. (2014), hoạt động tích chập được xác định trong tầng Fourier bằng cách tính phân tích độ riêng của ma trận Laplacian của đồ thị, dẫn đến việc tính toán có thể nặng và bộ lọc không phần tử tọa độ. Những vấn đề này đã được giải quyết bởi các công trình sau. Henaff et al. (2015) đã giới thiệu một tham số hóa của bộ lọc phần tử với các hệ số mượt để làm cho chúng có tọa độ phần tử. Sau đó, Defferrard et al. (2016) đề xuất sử dụng phép mở rộng Chebyshev của ma trận Laplacian để ước lượng bộ lọc, loại bỏ việc tính toán các vector riêng của Laplacian và cho ra các bộ lọc có tính địa lý, Kipf \& Welling (2017) đã đơn giản hoá phương pháp trước bằng cách hạn chế bộ lọc hoạt động trong một vùng lân cận xung quanh mỗi nút. Tuy nhiên, trong tất cả các phương pháp đồng bộ hoá tần số trên, các bộ lọc được học phụ thuộc vào cơ sở vector riêng của Laplacian, điều này phụ thuộc vào cấu trúc đồ thị. Do đó, một mô hình được huấn luyện trên một cấu trúc cụ thể không thể được áp dụng trực tiếp cho một đồ thị với cấu trúc khác.

Trên một bên, các phương pháp không phổ hoạt (Duvenaud et al., 2015; Atwood \& Towsley,2016; Hamilton et al., 2017) xác định việc tích chập trực tiếp trên đồ thị, hoạt động trên nhóm các đỉnh gần về vị trí. Một trong những thách thức của các phương pháp này là xác định một toán tử hoạt động với các lân cận có kích thước khác nhau và giữ thuộc tính chia sẻ trọng lượng của CNN. Trong một số trường hợp, điều này yêu cầu học một ma trận trọng lượng cụ thể cho mỗi bậc đỉnh (Duvenaud et al., 2015), sử dụng các lũy thừa của ma trận chuyển đổi để xác định lân cận và học trọng lượng cho mỗi kênh đầu vào và bậc lân cận (Atwood \& Towsley, 2016), hoặc trích xuất và chuẩn hóa các lân cận chứa số lượng cố định của đỉnh (Niepert et al., 2016). Monti et al. (2016) đã trình bày mô hình tổ hợp CNN (MoNet), một phương pháp vị trí cho phép tổng quát hóa kiến trúc CNN sang đồ thị. Gần đây, Hamilton et al. (2017) đã giới thiệu GraphSAGE, Một phương pháp tính biểu diễn của nút theo cách tuân theo. Kỹ thuật này hoạt động bằng cách mẫu một khu vực cố định kích thước của mỗi nút, sau đó thực hiện một trọng tải chung cho nó (như là trung bình của tất cả các vectơ đặc trưng của hàng xóm được mẫu, hoặc kết quả của việc cho chúng qua mạng nơ-ron tần suất). Phương pháp này đã cho kết quả tuyệt vời trên nhiều tiêu chuẩn thu hồi lớn.


Cơ chế tập trung đã trở thành gần như một tiêu chuẩn trong nhiều tác vụ dựa trên chuỗi (Bahdanau et al., 2015; Gehring et al., 2016). Một trong những lợi ích của các mạng tập trung là cho phép xử lý với kích thước đầu vào biến đổi, tập trung vào các phần quan trọng nhất của đầu vào để thực hiện quyết định. Khi sử dụng mạng tập trung để tính toán một biểu diễn của một chuỗi, nó thường được gọi là tập trung tự hoặc tập trung trong. Cùng với mạng nơ-ron lặp lại (RNNs) hoặc tích chập, tập trung tự đã chứng minh là hữu ích cho các tác vụ như đọc máy (Cheng et al., 2016) và học biểu diễn câu (Lin et al., 2017). Tuy nhiên, Vaswani et al. (2017) cho thấy rằng tập trung tự không chỉ có thể cải thiện một phương pháp dựa trên RNNs hoặc tích chập, mà còn đủ để xây dựng một mô hình mạnh mẽ nhận được hiệu năng tiên tiến nhất trong tác vụ dịch máy.

Lấy cảm hứng từ các công trình mới nhất, chúng tôi giới thiệu một kiến trúc dựa trên chú ý để thực hiện phân loại nút của dữ liệu cấu trúc đồ thị. Ý tưởng là tính toán biểu diễn ẩn của mỗi nút trong đồ thị, bằng cách chú ý trên các nút lân cận của nó, theo một chiến lược chú ý tự. Kiến trúc chú ý có một số thuộc tính quan trọng: (1) hoạt động hiệu quả, vì nó có thể được paralllel hoá giữa các cặp nút lân cận; (2) nó có thể được áp dụng cho các nút đồ thị có số bậc khác nhau bằng cách chỉ định các trọng lượng tùy ý cho các nút lân cận; và (3) mô hình có thể trực tiếp được áp dụng cho các vấn đề học theo một cách dựa trên thực tế, Bao gồm các nhiệm vụ mà mô hình phải tổng hợp đến các đồ thị hoàn toàn chưa từng thấy. Chúng tôi xác nhận giải pháp được đề xuất trên bốn mốc thử nghiệm khó: mạng trích dẫn Cora, Citeseer và Pubmed cũng như tập dữ liệu tương tác protein-protein theo phương pháp chỉ dẫn, đạt được hoặc phù hợp với kết quả tiên tiến nhất trong lĩnh vực để nổi bật tiềm năng của các mô hình dựa trên chú ý khi đối mặt với các đồ thị tùy ý.

Nên chú ý rằng, giống như Kipf \& Welling (2017) và Atwood \& Towsley (2016), công trình của chúng tôi cũng có thể được tái cấu trúc như một trường hợp cụ thể của MoNet (Monti et al., 2016). Hơn nữa, cách tiếp cận chúng tôi chia sẻ tính toán mạng nơ-ron qua các cạnh giống như cách tạo dạng của các mạng quan hệ (Santoro et al., 2017) và VAIN (Hoshen, 2017), trong đó các mối quan hệ giữa các đối tượng hoặc đại diện được tổng hợp theo cặp, bằng cách sử dụng một mục đích chung. Tương tự, mô hình chú ý được đề xuất của chúng tôi có thể được kết nối với các công trình của Duan et al. (2017) và Denil et al. (2017), sử dụng một hoạt động chú ý lân cận để tính toán các hệ số chú ý giữa các đối tượng khác nhau trong một môi trường. Các phương pháp liên quan khác bao gồm nhúng tính linearly cục bộ (LLE) (Roweis \& Saul, 2000) và các mạng nhớ (Weston et al., 2014). LLE chọn một số cố định của các láng giềng xung quanh mỗi điểm dữ liệu,và học một trọng số trọng tâm cho mỗi điểm lân cận để tái tạo mỗi điểm như là tổng trọng tâm của các điểm lân cận của nó. Bước tối ưu hóa thứ hai trích xuất nhúng đặc trưng của điểm. Mạng nhớ cũng chia sẻ một số kết nối với công việc của chúng tôi, đặc biệt là nếu chúng tôi hiểu khu vực xung quanh một nút như là bộ nhớ, được sử dụng để tính toán các đặc trưng của nút bằng cách chú ý về giá trị của nó, và sau đó được cập nhật bằng cách lưu trữ các đặc trưng mới tại cùng vị trí 



